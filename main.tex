
\documentclass[10pt, letterpaper]{article}
\usepackage[utf8]{inputenc}
\usepackage[T1]{fontenc}
\usepackage{geometry}
\geometry{margin=1in}
\usepackage{enumitem}
\usepackage{hyperref}
\usepackage{titlesec}
\usepackage{lmodern}
\usepackage{fontawesome5}
\usepackage{xcolor}

\titleformat{\section}{\Large\bfseries\color{blue}}{}{0em}{}[\titlerule]
\pagestyle{empty}

\begin{document}

\begin{center}
    {\Huge \textbf{Ryan Matthews}}\\[0.2cm]
    Munich, Germany \\
    \href{mailto:ryanjmatthews@gmail.com}{ryanjmatthews@gmail.com} ~|~ 
    +49 176 34599887 ~|~ 
    \href{https://linkedin.com/in/yourlinkedin}{LinkedIn} ~|~ 
    \href{https://github.com/ryanmatthews}{GitHub}
\end{center}

\vspace{0.5cm}

\section*{Work Experience}

\textbf{Luminar} \hfill Munich \\
\textit{System Architect, ADAS/AD} \hfill Jul 2022 -- Present
\begin{itemize}[leftmargin=1.5em]
    \item Defined technical direction across software, SoC, and Lidar.
    \item Pivoted to Lidar fallback perception for AI-centric OEMs.
    \item Introduced modular SoC architecture for late-cycle vehicle integration.
\end{itemize}

\textit{Product Owner, Active Safety} \hfill Sep 2020 -- Jun 2022
\begin{itemize}[leftmargin=1.5em]
    \item Led NCAP 2023-compliant Lidar-only safety stack.
    \item Presented at CES 2022 \& 2023.
    \item Built and mentored the ADAS dev team.
\end{itemize}

\vspace{0.3cm}

\section*{Education}
\textbf{B.Eng (Hons), Mechatronic Engineering}, University of Adelaide\\
First Class Honours, Dean's List, Aerospace Project Award

\textbf{Bachelor of Economics}, University of Adelaide\\
Major in mathematical modelling and statistics

\vspace{0.3cm}

\section*{Skills}
C++, Python, Model-Based Design, ISO 26262, AUTOSAR, Agile Product Ownership

\vspace{0.3cm}

\section*{Languages}
English (Native), German (Full Professional)

\end{document}
